\section{并行评估:Transformer vs. GNN vs. Classical}

\subsection{评估流程}
评估脚本读取 \texttt{train\_data\_final.npz} 中的样本,分别对 Transformer 与 GNN 前向得到 $\beta$ 序列,并在同一 $H_p,H_d$ 下重放演化得到 $\mathrm{Cut}_{\mathrm{AI}}$;再用样本内保存的教师能量(或缺失时重新跑教师)得到 $\mathrm{Cut}_{\mathrm{FALQON}}$,计算 $\mathrm{AR}$。

\subsection{Slurm 数组并行与结果汇总}
在 CPU 节点上,评估采用数组任务把样本区间分成若干块,每块内部使用多进程池并行计算,并将每块结果保存为 \texttt{output/ar\_parts/part\_\{start\}.npy}(GNN 额外保存 \texttt{part\_\{start\}\_gnn.npy})。

最终用合并脚本计算全局均值与标准差:
\begin{lstlisting}
python3 scripts/merge_ar_parts.py --parts_dir output/ar_parts --kind transformer
python3 scripts/merge_ar_parts.py --parts_dir output/ar_parts --kind gnn
\end{lstlisting}

\subsection{实验统计结果与讨论}
表 \ref{tab:dataset_results} 汇总了当前项目流水线在训练/评估数据集上的统计口径。需要强调:由于训练数据分布为 4--10 节点随机图,若要讨论 12/20 节点外推,应另行生成对应规模的数据集并重复相同评估流程。

\begin{table}[htbp]
\centering
\caption{训练/评估数据集(4--10 节点随机图)上的 AR 统计结果}
\label{tab:dataset_results}
\begin{tabular}{lcccc}
\toprule
测试场景 & 样本数 & 平均近似比 (Avg AR) & 标准差 (Std) & 备注 \\
\midrule
训练/评估数据集 (4--10 节点) & 744 & 1.0132 & 0.4834 & Transformer Aggregated \\
GNN (Baseline) & 5 & 0.9957 & 0.0141 & 仅含部分分片示例 \\
\bottomrule
\end{tabular}
\end{table}
