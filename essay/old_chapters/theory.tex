\section{理论推导与算法框架}

\subsection{反馈动力学的李雅普诺夫导出}
考虑由问题哈密顿量 $H_p$ 和驱动哈密顿量 $H_d$ 描述的系统。目标是最小化成本函数 $C(t)=\langle\psi(t)|H_p|\psi(t)\rangle$。根据薛定谔方程,其对时间的导数为:
\begin{equation}
\frac{d C(t)}{dt} = i \beta(t) \langle \psi(t) | [H_d, H_p] | \psi(t) \rangle.
\end{equation}
为了确保成本函数随时间单调递减(即 $dC/dt\le 0$),构造反馈控制律:
\begin{equation}
\beta(t) = -\alpha\,\langle \psi(t) | i[H_d, H_p] | \psi(t) \rangle.
\end{equation}
在 FALQON 的离散化实现中,状态演化算符为 $U_p=e^{-i\beta_p H_d}e^{-iH_p\Delta t}$。

\paragraph{与实现的一致性} 项目实现中首先预计算 $A=i(H_d H_p - H_p H_d)$(见算法文件),并在每层更新
\[
\beta_p = -\alpha\,\langle \psi_p|A|\psi_p\rangle,\quad \psi_{p+1}=e^{-i\beta_p H_d}e^{-iH_p}\,\psi_p.
\]
这使得我们可以把“反馈生成的 $\beta$ 序列”看作一个监督信号,用于训练学生模型回归整段轨迹。

\subsection{FALQON 的动力学方程组 (ODE)}
将反馈律代回薛定谔方程,我们可以得到一个闭环的、非线性的自主微分方程组:
\begin{equation}
\frac{d}{dt} |\psi(t)\rangle = -i \left( H_P - \alpha \langle \psi(t) | i[H_d, H_P] | \psi(t) \rangle H_d \right) |\psi(t)\rangle
\end{equation}
这个方程揭示了 FALQON 参数序列的本质:
\begin{itemize}
	\item \textbf{非线性性}:哈密顿量不再是外加的,而是依赖于状态 $|\psi(t)\rangle$ 本身。
	\item \textbf{动力学轨迹}:我们在离散算法中看到的参数序列 $\vec{\beta}$,实际上是对连续函数 $\beta(t)$ 的时间序列采样。
	\item \textbf{初值依赖性}:整个轨迹完全由初始状态 $|\psi_0\rangle$ 和系统的哈密顿量 $H_P$ 决定。
\end{itemize}

\subsection{库普曼算子理论与序列可预测性}
为了解释为何 Transformer 能有效预测该序列,我们引入库普曼算子(Koopman Operator)理论。对于非线性动力系统,库普曼算子 $\mathcal{K}$ 定义为作用在可观测量函数 $g$ 上的线性算子:$\mathcal{K} g(x_t) = g(x_{t+1})$。
在 FALQON 中,控制参数 $\beta_t$ 本身就是一个可观测量:
\begin{equation}
\beta_t = g(|\psi_t\rangle) = -\alpha \langle \psi_t | i[H_d, H_P] | \psi_t \rangle
\end{equation}
根据库普曼模态分解定理,$\beta_t$ 可以展开为:
\begin{equation}
\beta_t \approx \sum_{k=1}^K a_k(H_P) \cdot \lambda_k^t
\end{equation}
这表明,参数时间序列本质上是由一组特定的本征频率($\lambda_k$)控制的线性叠加。Transformer 的自注意力机制实际上充当了谱滤波器(Spectral Filter)的角色,从输入的图嵌入中提取出驱动动力学演化的关键模态系数 $a_k(H_P)$,从而实现对整个轨迹的精确重构。

\subsection{图结构的哈密顿量编码}
对于 $n$ 节点的 MaxCut 问题,问题哈密顿量定义为:
\begin{equation}
H_p = \sum_{(i,j) \in E} w_{ij} \frac{I - Z_i Z_j}{2}.
\end{equation}
由于 $H_p$ 完全由图的邻接矩阵 $A$ 决定,参数轨迹 $\vec{\beta}$ 是矩阵 $A$ 的非线性函数。本文假设存在映射 $\mathcal{F}:A\mapsto\vec{\beta}$,并利用神经网络进行逼近。

\paragraph{Cut 与能量的换算} 在代码评估中,我们以能量期望 $E=\langle H_p\rangle$ 换算切割值:
\begin{equation}
\mathrm{Cut} = \frac{|E(G)| - 2E}{2},\qquad \mathrm{AR}=\frac{\mathrm{Cut}_{\mathrm{AI}}}{\mathrm{Cut}_{\mathrm{FALQON}}}.
\end{equation}
