\section{Transformer 模型架构与创新点}

\subsection{输入特征嵌入与谱位置编码}
模型将展平后的邻接矩阵作为 Token 输入。为使模型感知拓扑,我们引入了归一化拉普拉斯算子 $L = I - D^{-1/2} A D^{-1/2}$ 的特征分解,提取其特征向量作为谱位置编码(Spectral Positional Encoding)。这使得模型能够直接学习与哈密顿量谱统计相关的结构特征。

\subsection{编码器结构与注意力机制}
核心组件为多头注意力机制(Multi-Head Attention):
\begin{equation}
\text{Attention}(Q, K, V) = \text{softmax}\left( \frac{QK^T}{\sqrt{d_k}} \right) V.
\end{equation}
\textbf{物理直觉:}全局注意力机制允许模型同时分析图中的局部簇与长程连通性。

\subsection{预测头与输出}
编码后的图级表示可通过池化/读出汇聚为向量,并经由 MLP 直接回归长度 $P=30$ 的连续参数序列。该“整段轨迹回归”对应一次前向推理输出全层参数,从而在推理阶段避免逐层反馈测量。

\subsection{方法论分析:为何 Transfomer 优于 GNN 捕捉宏观谱统计}
图神经网络(GNN)通常基于消息传递机制(MPNN),其 $L$ 层聚合仅能覆盖 $L$-跳邻域。若要捕捉决定 FALQON 演化轨迹的全局谱特征(如 $H_p$ 的高阶矩 $\text{Tr}(A^k)$ 或特征值分布),MPNN 需要堆叠至图直径深度,这常导致过平滑(Over-smoothing)问题,使得节点表示趋同而丢失结构信息:
\begin{equation}
\lim_{L \to \infty} \mathbf{H}^{(L)} \approx \mathbf{1} \mathbf{c}^T.
\end{equation}
相比之下,Sequence Transformer 具有全局感受野(Global Receptive Field)。通过自注意力机制:
\begin{equation}
\text{Attn}(\mathbf{X})_i = \sum_{j=1}^N \frac{\exp(\mathbf{x}_i^T \mathbf{x}_j / \sqrt{d_k})}{\sum_{k=1}^N \exp(\mathbf{x}_i^T \mathbf{x}_k / \sqrt{d_k})} \mathbf{v}_j,
\end{equation}
任意节点对 $(i, j)$ 无论拓扑距离多远,其交互路径长度均为 1。这种完全图式的连接使其能直接聚合全图信息来逼近全局谱统计量(Global Spectral Statistics),从而更准确地建立 $A(G) \to \vec{\beta}$ 的非线性映射,避免了局部算子在重构长程关联时的指数级衰减。
