\section{物理分析:泛化能力与跨规模稳健性}

\subsection{谱密度与泛化基础}
从物理直觉上,FALQON 的反馈律由对易子期望值驱动,而对易子的统计行为与 $H_p$(由图结构决定)的谱性质密切相关。对于 Erd\H{o}s--R\'{e}nyi 随机图,随着规模增大,其邻接/拉普拉斯谱分布在统计意义上趋于稳定;因此我们提出一种可检验的假说:\emph{当谱统计在不同规模间保持相近时,学生模型更可能实现跨规模泛化}。

\subsection{近似比分析与 12 节点外推示例}
在一个 12 节点外推测试样本集(示例:100 个样本)上,我们观察到 Transformer 的 Avg AR 为 1.0031,Std 为 0.0376(表 \ref{tab:extrap_12_results})。需要强调:外推结论需要在固定规模的独立测试集上按同一口径复现实验。

\begin{table}[htbp]
\centering
\caption{示例:Transformer 在训练域内与 12 节点外推测试上的 AR 统计(含 Min/Max)}
\label{tab:extrap_12_results}
\begin{tabular}{lccccc}
\toprule
测试场景 & 样本数 & 平均近似比 (Avg AR) & 标准差 (Std) & 最小值 (Min) & 最大值 (Max) \\
\midrule
训练域内 (4--10 节点) & 100 & 1.0009 & 0.1411 & 0.6147 & 1.4577 \\
外推测试 (12 节点) & 100 & \textbf{1.0031} & \textbf{0.0376} & \textbf{0.8181} & \textbf{1.1080} \\
\bottomrule
\end{tabular}
\end{table}

\paragraph{Transformer vs. GNN 对比(同口径 AR,分片示例)}
为了给出与局部消息传递基线的直接对照,我们在同一评估口径下统计了 Transformer、GNN 与经典 FALQON 的近似比(AR)。在一次评估分片上得到的统计量为:Transformer 的 Avg AR 为 1.025287、Std 为 0.070539;GNN 的 Avg AR 为 0.995718、Std 为 0.014080;经典 FALQON 作为基准按定义为 1.0。

\begin{table}[htbp]
\centering
\caption{Transformer / GNN / Classical 的近似比(AR)对比(单次评估分片统计)}
\label{tab:ar_compare_part}
\begin{tabular}{lcc}
\toprule
方法 & 平均近似比 (Avg AR) & 标准差 (Std) \\
\midrule
Transformer & 1.025287 & 0.070539 \\
GNN (Baseline) & 0.995718 & 0.014080 \\
Classical FALQON & 1.000000 & 0.000000 \\
\bottomrule
\end{tabular}
\end{table}

\textbf{备注:}上述为“分片”统计结果;若使用数组任务跑完整测试集,应对所有分片合并后再报告全局均值与方差。

\subsection{大规模随机正则图上的预测曲线聚类现象}
为了在 $N>20$ 的规模上验证“预测参数曲线的聚类(common profile)”现象,我们在随机 $d$-正则图分布上进行零次推理测试,并仅分析学生模型输出的 $\beta$ 序列形状统计(此处不再运行经典 Teacher 的全量 statevector 反馈仿真,以避免 $N$ 增大带来的指数资源开销)。

实验设置为:固定节点数 $N=24$、度数 $d=3$,随机采样 3-正则图,对每个图预测长度 $P=30$ 的参数序列 $\vec{\beta}$。随后对这些曲线做层次聚类,并用 PCA 将曲线嵌入到二维平面观察聚类结构。

\begin{figure}[htbp]
    \centering
    \includegraphics[width=0.95\textwidth]{picture/regular_clustering_curves_zoom.png}
    \caption{随机 3-正则图($N=24$)上预测的 $\beta$ 曲线聚类结果。图中展示各簇均值曲线(为突出细微差异,绘图时对早期极端尖峰步骤做了缩放/截取)。可见在固定分布下,预测曲线会集中到少数几类“共同轮廓”。}
    \label{fig:regular_curve_cluster}
\end{figure}

\begin{figure}[htbp]
    \centering
    \includegraphics[width=0.65\textwidth]{picture/regular_clustering_pca.png}
    \caption{对预测的 $\beta$ 曲线做 PCA 降维后的散点图(颜色表示聚类标签)。聚类在低维嵌入空间中依然可分,说明曲线差异具有低维结构。}
    \label{fig:regular_curve_pca}
\end{figure}

\paragraph{理论洞察:Kesten-McKay 定律与轨迹普适性}
参数曲线出现聚类(Common Profile)现象的根本原因在于图谱统计的收敛性。FALQON 的每步反馈值 $\beta_p \propto \langle [H_d, H_p] \rangle$ 非线性地依赖于哈密顿量 $H_p$ 的各阶谱矩(Spectral Moments) $\mu_k = \frac{1}{N}\text{Tr}(H_p^k) = \int \lambda^k \rho(\lambda) d\lambda$。

根据随机矩阵理论,对于随机 $d$-正则图,当 $N \to \infty$ 时,其邻接矩阵特征值的经验谱密度 $\rho_N(\lambda)$ 依概率弱收敛于 Kesten-McKay 分布(亦称作相关随机游走谱分布)\cite{ref10,ref11}:
\begin{equation}
\rho_{\text{KM}}(\lambda) = \begin{cases} 
\frac{d \sqrt{4(d-1) - \lambda^2}}{2\pi (d^2 - \lambda^2)} & |\lambda| \le 2\sqrt{d-1}, \\
0 & \text{otherwise}.
\end{cases}
\label{eq:kesten_mckay}
\end{equation}
这解释了图 \ref{fig:regular_curve_cluster} 中的聚类现象:尽管具体的图实例 $G$ 不同,但它们在热力学极限($N \gg 1$)下共享相同的极限谱分布 $\rho_{\text{KM}}$。既然控制动力学的 $\beta$ 序列是谱分布的泛函 $\vec{\beta} = \mathcal{G}[\rho(\lambda)]$,那么谱分布的普适性(Universality)必然导致控制轨迹的普适性。

\begin{figure}[htbp]
    \centering
    \includegraphics[width=0.75\textwidth]{picture/regular_clustering_spectrum.png}
    \caption{随机 3-正则图($N=24$)的缩放邻接矩阵谱密度(去除最大特征值后)与 Wigner 半圆律参考曲线的对照。谱密度在 bulk 区域呈稳定形态,为“预测曲线聚类”的随机矩阵解释提供直觉支撑。}
    \label{fig:regular_spectrum_semicircle}
\end{figure}

\begin{figure}[htbp]
    \centering
    \includegraphics[width=0.8\textwidth]{picture/prediction_result.png}
    \caption{预测的 $\beta$ 参数序列(红虚线)与经典 FALQON 真实序列(蓝实线)的对比轨迹。}
    \label{fig:prediction_result}
\end{figure}
