\section{模型实现与训练细节}

\subsection{Transformer 学生:全局建模}

为了体现“时间序列”和“图结构”的双重特性,我们设计了“谱-时序 Transformer”(Spectral-Temporal Transformer)。该架构包含两个核心模块:

\begin{enumerate}
	\item \textbf{谱编码器(Spectral Encoder)}:输入图的拉普拉斯特征值与特征向量。为了解决特征向量的符号模糊性(Sign Ambiguity),我们采用了符号不变网络(SignNet)进行预处理,输出图的全局谱嵌入 $h_{graph}$。
	\item \textbf{时序解码器(Temporal Decoder)}:采用 Decoder-Only 结构。输入时间步索引 $t$ 的正弦位置编码作为 Query,通过交叉注意力(Cross-Attention)机制查询 $h_{graph}$。这模拟了库普曼算子根据当前时间步调用特定谱模态的过程,从而生成时刻 $t$ 的控制参数 $\beta_t$。
\end{enumerate}

此外,为了解决大规模数据生成的瓶颈,我们引入了物理信息微调策略。在小规模图上进行预训练后,在大规模图上直接利用 FALQON 的动力学方程构建物理损失函数 $\mathcal{L}_{Physics} = \langle \psi_P(\hat{\beta}) | H_P | \psi_P(\hat{\beta}) \rangle$,通过可微量子模拟器进行自监督更新。

\subsection{GNN 基线:局部消息传递}
为提供对照,我们实现了一个不依赖外部图库的轻量 GNN:输入为 padding 后的邻接矩阵与 mask;加入 self-loop 并做 $\tilde{A}=D^{-1/2}(A+I)D^{-1/2}$;消息传递为 $H^{(\ell+1)}=\sigma(\tilde{A}H^{(\ell)}W^{(\ell)})$;读出采用 mask 加权的 mean pooling + MLP 输出 $\beta$ 序列。
