\section{充实内容与创新点建议:架构演进路线}
针对进一步充实内容的需求,本章节提出了四个具体的、高价值的创新方向,旨在解决当前的理论隐患并将研究推向该领域的前沿。

\subsection{创新点一:引入符号不变网络(SignNet/BasisNet)}
\textbf{痛点解决}:彻底解决使用拉普拉斯特征向量时存在的谱特征符号歧义(Sign Ambiguity)与基底歧义(Basis Ambiguity)问题。

\textbf{具体方案}:
不直接将特征向量 $V$ 输入 Transformer,而是通过一个\textbf{符号不变编码器(Sign-Invariant Encoder)}进行预处理。根据 Lim 等人 (2023) 的工作\cite{ref8},可以构建如下映射:
\begin{equation}
f(V) = \rho \left( \sum_{i} \phi(v_i) \right)
\end{equation}
其中 $\phi$ 是一个点级(Point-wise)神经网络(如 MLP),设计为偶函数以满足 $f(v) = f(-v)$。

\textbf{效果}:无论线性代数求解器输出的特征向量符号是 $v$ 还是 $-v$,编码器的输出完全一致。将此模块作为 Transformer 的 Tokenizer,将显著降低训练 Loss 的方差,并提升在同构图上的测试稳定性。

\subsection{创新点二:物理信息驱动的无监督微调(Physics-Informed Fine-tuning)}
\textbf{痛点解决}:目前模型依赖于经典 FALQON 生成的标签(Ground Truth),生成这些标签极其耗时(需要全波函数模拟 $O(2^N)$),限制了训练集只能覆盖小图。

\textbf{具体方案}:利用\textbf{可微量子模拟器}(如 PennyLane 或 JAX-Quantum),构建一个无需教师标签的物理损失函数:
\begin{equation}
\mathcal{L}(\theta) = \langle \psi(\beta_\theta) | H_P | \psi(\beta_\theta) \rangle
\end{equation}
其中 $\beta_\theta$ 是学生网络的输出。
\textbf{流程}:
\begin{enumerate}
    \item \textbf{预训练(Pre-training)}:在小图($N \le 12$)上使用有监督学习(MSE Loss),让模型快速学会 FALQON 的基本轨迹形状。
    \item \textbf{微调(Fine-tuning)}:在大图($N > 20$)上,不再运行经典 FALQON,而是直接通过可微模拟器计算能量并对网络参数进行梯度下降。
\end{enumerate}
\textbf{价值}:这将使模型能够探索出比贪婪 FALQON \textbf{更优}的轨迹(AR $> 1.0$ 的现象将不再是偶然,而是目标),并突破训练数据生成的算力瓶颈。

\subsection{创新点三:通用谱控制器(Universal Spectral Controller)}
\textbf{痛点解决}:处理大规模图($N \gg 1000$)时,即便是推理过程,计算特征分解($O(N^3)$)也可能过于昂贵。

\textbf{具体方案}:基于 Kesten-McKay 理论,既然参数主要由谱矩(Spectral Moments)决定,我们可以设计一个极其轻量级的 MLP 模型。
\begin{itemize}
    \item \textbf{输入}:仅输入图的前 $k$ 阶谱矩 $\mu_k = \text{Tr}(A^k)$(可以通过随机迹估计法在 $O(N)$ 时间内估算,无需特征分解)。
    \item \textbf{输出}:预测参数序列 $\vec{\beta}$。
\end{itemize}
\textbf{实验验证}:对比 Full Transformer 和 Moment-MLP 的性能。如果两者接近,则证明了“参数由谱统计量决定”的物理假设。

\subsection{创新点四:噪声感知的鲁棒推理(Noise-Resilient Inference)}
\textbf{痛点解决}:NISQ 设备充满噪声。经典的 FALQON 因为闭环反馈,会将测量噪声引入下一层的参数,导致误差累积(Random Walk Drift)。
\textbf{具体方案}:将学生模型重新定义为一种\textbf{量子误差缓解(QEM)}工具\cite{ref4}。
\begin{itemize}
    \item \textbf{训练策略}:在无噪声模拟环境下训练,目标是理想 FALQON 轨迹。
    \item \textbf{应用场景}:在有噪声的量子硬件上,不运行实时的 FALQON 反馈回路,而是直接应用学生模型预测的“理想参数”。
\end{itemize}
\textbf{预期结果}:由于学生模型是在无噪数据上训练的,它实施的是\textbf{开环控制(Open-Loop Control)},完全规避了硬件上的反馈测量噪声累积。实验有望证明:随着硬件噪声增加,学生模型的性能将显著优于标准 FALQON。
