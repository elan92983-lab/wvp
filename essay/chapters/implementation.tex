\section{模型实现与训练细节}

\subsection{Transformer 学生:全局建模}
项目主模型将展平后的邻接矩阵视为输入特征,通过线性层映射到 $d_{\text{model}}=64$ 的嵌入空间,并用多层 Transformer 解码器回归长度 $P=30$ 的 $\beta$ 序列。尽管代码中将“图特征”压缩为单 token 进行解码,这仍可视为一种全局表示学习:模型用注意力机制对整图统计进行建模,输出整段轨迹。

\subsection{GNN 基线:局部消息传递}
为提供对照,我们实现了一个不依赖外部图库的轻量 GNN:输入为 padding 后的邻接矩阵与 mask;加入 self-loop 并做 $\tilde{A}=D^{-1/2}(A+I)D^{-1/2}$;消息传递为 $H^{(\ell+1)}=\sigma(\tilde{A}H^{(\ell)}W^{(\ell)})$;读出采用 mask 加权的 mean pooling + MLP 输出 $\beta$ 序列。
