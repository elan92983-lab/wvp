\section{结论}

本文提出了一种基于谱-时序 Transformer 的 FALQON 参数预测方法,通过“教师-学生”框架实现了变分量子优化中反馈参数的零次推理生成。主要贡献和发现总结如下:

\subsection{主要贡献}
\begin{enumerate}
	\item \textbf{架构创新}:设计了结合图谱编码(SignNet)与时序解码(Transformer Decoder)的混合架构,首次将拉普拉斯谱的符号不变性引入量子优化参数预测任务。
    
	\item \textbf{训练策略}:提出了基于 Scheduled Sampling 的自回归训练方法,有效缓解了训练-推理不一致问题。
    
	\item \textbf{系统性实验}:在包含约 1000 个随机图样本的数据集上进行了全面评估,并首次按动力学特性(收敛型/振荡型)对样本进行分类分析。
\end{enumerate}

\subsection{核心发现}
\begin{enumerate}
	\item 模型在\textbf{收敛型样本}上保持较高一致性,平均相关系数约 0.813,最佳可达 0.990,证明了神经网络预测 FALQON 参数的可行性。
    
	\item 在\textbf{振荡型样本}上,平均相关系数约 0.67,虽存在达到 0.99 的个例,但也暴露出低相关度的困难样本,显示对混沌动力学仍有改进空间。
    
	\item 模型在所有样本上均能准确拟合决定优化方向的\textbf{初始峰-谷结构}(Layer 0-5),这对实际应用至关重要。
\end{enumerate}

\subsection{实践意义}
本方法的核心价值在于:用一次神经网络前向传播替代 $O(P^2)$ 累积深度的量子测量,从而在 NISQ 设备上显著降低测量开销和噪声累积。对于能量景观具有明显全局最小值的优化问题(如大部分 Erdős-Rényi 随机图上的 MaxCut),本方法可直接部署使用。

\subsection{未来展望}
后续工作将聚焦于三个方向:
\begin{enumerate}
	\item 在固定 $N=12, 20$ 的外推数据集上进行同口径对比,验证跨规模泛化能力;
	\item 探索混合预测策略,结合神经网络与物理约束处理振荡型样本;
	\item 在更贴近 NISQ 的噪声模型下评估“零次推理”对采样开销的真实收益。
\end{enumerate}

\begin{thebibliography}{99}

\bibitem{ref1}
arXiv:2405.00781v2 [quant-ph] 8 May 2025,
\url{https://arxiv.org/pdf/2405.00781}



\bibitem{ref2}
Robust Feedback-Based Quantum Optimization: Analysis of Coherent Control Errors. IEEE Xplore,
\url{https://ieeexplore.ieee.org/iel8/11157924/11157953/11158422.pdf}

\bibitem{ref3}
Feedback-Based Quantum Optimization. ResearchGate,
\url{https://www.researchgate.net/publication/366240286_Feedback-Based_Quantum_Optimization}

\bibitem{ref4}
Adaptive Sampling Noise Mitigation Technique for Feedback-based Quantum Algorithms,
\url{https://www.iccs-meeting.org/archive/iccs2024/papers/148370309.pdf}

\bibitem{ref5}
Robust feedback-based quantum optimization: analysis of coherent control errors,
\url{https://www.researchgate.net/publication/393379559_Robust_feedback-based_quantum_optimization_analysis_of_coherent_control_errors}

\bibitem{ref6}
Extending QAOA-GPT to Higher-Order Quantum Optimization Problems. arXiv,
\url{https://arxiv.org/html/2511.07391v1}

\bibitem{ref7}
QAOA Parameter Transferability for Maximum Independent Set using Graph Attention Networks. ResearchGate,
\url{https://www.researchgate.net/publication/391329610_QAOA_Parameter_Transferability_for_Maximum_Independent_Set_using_Graph_Attention_Networks}

\bibitem{ref8}
Laplacian Canonization: A Minimalist Approach to Sign and Basis Invariant Spectral Embedding. NeurIPS,
\url{https://proceedings.neurips.cc/paper_files/paper/2023/file/257b3a7438b1f3709e91a86adf2fdc0a-Paper-Conference.pdf}

\bibitem{ref9}
Sign and Basis Invariant Networks for Spectral Graph Representation Learning. ICLR 2026,
\url{https://iclr.cc/virtual/2022/8714}

\bibitem{ref10}
Short Cycles in Random Regular Graphs. ResearchGate,
\url{https://www.researchgate.net/publication/220342833_Short_Cycles_in_Random_Regular_Graphs}

\bibitem{ref11}
Edge rigidity and universality of random regular graphs of intermediate degree,
\url{https://www.unige.ch/~knowles/rrg_edge.pdf}

\bibitem{ref12}
Transferability of optimal QAOA parameters between random graphs,
\url{https://www.computer.org/csdl/proceedings-article/qce/2021/169100a171/1yEZ9MWWjSg}

\bibitem{ref13}
Robust feedback-based quantum optimization: analysis of coherent control errors. arXiv,
\url{https://arxiv.org/pdf/2507.02532}

\end{thebibliography}
