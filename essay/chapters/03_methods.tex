% 03_methods.tex
% 方法部分

本节详细介绍谱-时序 Transformer 的架构设计和训练策略。

\subsection{问题形式化}
\label{subsec:problem_formulation}

给定图 $G$ 的拉普拉斯谱 $(\Lambda, U)$,目标是预测 FALQON 参数序列 $\bm{\beta} = (\beta_0, \beta_1, \, \ldots, \, \beta_{P-1})$。我们将此建模为条件序列生成问题:
\begin{equation}
p(\bm{\beta} | G) = \prod_{t=0}^{P-1} p(\beta_t | \beta_{<t}, G)
\end{equation}

\subsection{模型架构}
\label{subsec:architecture}

如图 \ref{fig:architecture} 所示,谱-时序 Transformer 包含三个核心模块。

\subsubsection{符号不变谱编码器(SignNet)}
\label{subsec:signnet}

为解决特征向量的符号模糊性,我们采用 SignNet 进行预处理:
\begin{equation}
h_i = \rho\left( \phi(u_i) + \phi(-u_i) \right)
\end{equation}
其中 $\phi: \mathbb{R}^n \to \mathbb{R}^d$ 为 MLP,$\rho: \mathbb{R}^d \to \mathbb{R}^d$ 为聚合层。该设计保证 $h_i = h_{-i}$,消除符号歧义。

特征值通过独立的 MLP 编码后与 SignNet 输出融合:
\begin{equation}
m_i = \text{Fusion}\left( \text{MLP}(\lambda_i), h_i \right)
\end{equation}
得到 $M$ 个谱模态的表示 $\{m_i\}_{i=1}^M$,作为 Transformer 解码器的 Memory。

\subsubsection{时序解码器}

采用标准 Transformer Decoder \cite{vaswani2017attention},Query 由时间步的正弦位置编码和前一步预测值的嵌入组成:
\begin{equation}
q_t = \text{PE}(t) + \text{Embed}(\beta_{t-1}) + e_{\text{query}}
\end{equation}
其中 $e_{\text{query}}$ 为可学习的查询嵌入。

解码器通过交叉注意力查询谱模态信息:
\begin{equation}
\text{CrossAttn}(Q, K, V) = \text{softmax}\left( \frac{QK^T}{\sqrt{d}} \right) V
\end{equation}
其中 $K, V$ 来自谱编码器输出,$Q$ 来自时序查询。

\subsubsection{输出头}

解码器输出经 MLP 映射为标量预测:
\begin{equation}
\hat{\beta}_t = \text{MLP}(\text{Decoder}(q_t, M))
\end{equation}

\subsection{训练策略}
\label{subsec:training}

\subsubsection{Scheduled Sampling}

为缓解训练(使用真实 $\beta_{t-1}$)与推理(使用预测 $\hat{\beta}_{t-1}$)的不一致,我们采用 Scheduled Sampling \cite{bengio2015scheduled}:
\begin{equation}
\tilde{\beta}_{t-1} = \begin{cases}
\beta_{t-1}^{\text{true}} & \text{w.p. } 1 - \epsilon \\
\hat{\beta}_{t-1} & \text{w.p. } \epsilon
\end{cases}
\end{equation}
其中 $\epsilon$ 从 0 线性增长至 0.3。

\subsubsection{损失函数}

总损失由三部分组成:
\begin{equation}
\mathcal{L} = \mathcal{L}_{\text{MSE}} + \lambda_1 \mathcal{L}_{\text{temporal}} + \lambda_2 \mathcal{L}_{\text{tail}}
\end{equation}

\begin{itemize}
    \item \textbf{加权 MSE}:后段时间步赋予更高权重
    \begin{equation}
    \mathcal{L}_{\text{MSE}} = \frac{1}{P} \sum_{t=0}^{P-1} w_t (\hat{\beta}_t - \beta_t)^2, \quad w_t = 1 + \frac{t}{P}
    \end{equation}
    
    \item \textbf{时序梯度损失}:鼓励学习变化趋势
    \begin{equation}
    \mathcal{L}_{\text{temporal}} = \frac{1}{P-1} \sum_{t=1}^{P-1} \left( \Delta\hat{\beta}_t - \Delta\beta_t \right)^2
    \end{equation}
    
    \item \textbf{尾部方差损失}:约束后段的动态范围
\end{itemize}