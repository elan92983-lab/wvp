\section{引言与背景综述}
在嘈杂中型量子(NISQ)时代,变分量子算法(VQA)被广泛认为是通向实用量子优势的最可行路径之一。其中,量子近似优化算法(QAOA)在解决组合优化问题(如 MaxCut、MaxSAT)方面表现出了巨大的潜力。然而,QAOA 的实际部署面临着严峻的挑战:经典的参数优化循环。这一过程需要在高维、非凸的能量景观中寻找最优参数 $\bm{\theta}^*$,极易陷入局部极小值,且在大规模系统中面临“贫瘠高原”(Barren Plateau)现象,即梯度随量子比特数指数级消失。


虽然 FALQON 在理论上保证了收敛性并消除了参数搜索的需求,但它引入了新的计算瓶颈:测量开销(Measurement Overhead)。在 FALQON 的执行过程中,为了计算第 $k+1$ 层的参数 $\beta_{k+1}$,必须首先在量子计算机上制备出深度为 $k$ 的量子态,并测量对易子算符的期望值。对于一个深度为 $P$ 的电路,这一过程必须重复 $P$ 次,导致总累积深度达到 $O(P^2)$。此外,实时反馈还面临着严重的噪声累积问题:第 $k$ 层的测量误差会直接进入第 $k+1$ 层的参数,导致控制轨迹偏离理想的李雅普诺夫路径。

为了彻底打破这一瓶颈,本项目提出了一种“零次推理”(Zero-Shot Inference)的范式。不同于以往简单的回归视角,我们引入了“时间序列”与“动力系统”的核心观点。FALQON 的参数序列 $\beta_1, \beta_2, \dots$ 并非孤立的数值,而是系统状态在控制流形上随时间演化的轨迹。结合库普曼算子(Koopman Operator)理论 \cite{ref9},我们将这一非线性演化线性化,从而解释为何序列模型(如 Transformer)能够有效捕捉其动力学特征。同时,借鉴 PALQO 的物理信息学习框架 \cite{ref11},我们将 FALQON 的动力学方程直接作为损失函数,实现无需标签数据的自监督学习。

本文旨在对该研究成果进行阐述与评估。我们将从理论完备性、物理泛化机制、以及架构优化三个维度进行详尽剖析。首先介绍现有架构实现;随后深化理论基础,利用随机矩阵理论(Random Matrix Theory, RMT)中的 Kesten-McKay 分布解释参数聚类现象;最后讨论结合 SignNet、物理信息损失函数(Physics-Informed Loss)以及噪声感知训练的未来演进方向。
