\section{理论推导与算法框架}

\subsection{反馈动力学的李雅普诺夫导出}
考虑由问题哈密顿量 $H_p$ 和驱动哈密顿量 $H_d$ 描述的系统。目标是最小化成本函数 $C(t)=\langle\psi(t)|H_p|\psi(t)\rangle$。根据薛定谔方程,其对时间的导数为:
\begin{equation}
\frac{d C(t)}{dt} = i \beta(t) \langle \psi(t) | [H_d, H_p] | \psi(t) \rangle.
\end{equation}
为了确保成本函数随时间单调递减(即 $dC/dt\le 0$),构造反馈控制律:
\begin{equation}
\beta(t) = -\alpha\,\langle \psi(t) | i[H_d, H_p] | \psi(t) \rangle.
\end{equation}
在 FALQON 的离散化实现中,状态演化算符为 $U_p=e^{-i\beta_p H_d}e^{-iH_p\Delta t}$。

\paragraph{与实现的一致性} 项目实现中首先预计算 $A=i(H_d H_p - H_p H_d)$(见算法文件),并在每层更新
\[
\beta_p = -\alpha\,\langle \psi_p|A|\psi_p\rangle,\quad \psi_{p+1}=e^{-i\beta_p H_d}e^{-iH_p}\,\psi_p.
\]
这使得我们可以把“反馈生成的 $\beta$ 序列”看作一个监督信号,用于训练学生模型回归整段轨迹。

\subsection{图结构的哈密顿量编码}
对于 $n$ 节点的 MaxCut 问题,问题哈密顿量定义为:
\begin{equation}
H_p = \sum_{(i,j) \in E} w_{ij} \frac{I - Z_i Z_j}{2}.
\end{equation}
由于 $H_p$ 完全由图的邻接矩阵 $A$ 决定,参数轨迹 $\vec{\beta}$ 是矩阵 $A$ 的非线性函数。本文假设存在映射 $\mathcal{F}:A\mapsto\vec{\beta}$,并利用神经网络进行逼近。

\paragraph{Cut 与能量的换算} 在代码评估中,我们以能量期望 $E=\langle H_p\rangle$ 换算切割值:
\begin{equation}
\mathrm{Cut} = \frac{|E(G)| - 2E}{2},\qquad \mathrm{AR}=\frac{\mathrm{Cut}_{\mathrm{AI}}}{\mathrm{Cut}_{\mathrm{FALQON}}}.
\end{equation}
