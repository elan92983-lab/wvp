% 06_conclusion.tex
% 结论

本文提出了基于谱-时序 Transformer 的 FALQON 参数预测方法,通过"教师-学生"框架实现了量子优化参数的零次推理生成。主要发现包括:

\begin{enumerate}
    \item \textbf{可行性验证}:在收敛型样本上,模型达到 0.917 的平均相关系数,证明神经网络预测量子控制参数是可行的。
    \item \textbf{适用边界}:振荡型样本(主要来自正则图)的后段高频振荡难以准确预测,这是数据驱动方法面对混沌动力学的固有挑战。
    \item \textbf{理论保障}:从李雅普诺夫理论证明,只要预测误差不改变控制参数符号,系统仍能收敛。
\end{enumerate}

\paragraph{未来工作}
\begin{itemize}
    \item 在更大规模图($n > 20$)上验证跨规模泛化能力
    \item 探索混合策略:神经网络预测 + 物理约束细化
    \item 在真实量子硬件噪声模型下评估实际收益
\end{itemize}