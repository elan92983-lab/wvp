% 06_conclusion.tex
% 结论

本文提出了基于谱-时序 Transformer 的 FALQON 参数预测方法,通过"教师-学生"框架实现了量子优化参数的零次推理生成。我们不仅解决了电路深度维度的复杂度问题,更深入分析了量子比特数维度的可扩展性和噪声鲁棒性。

\subsection{主要贡献总结}

\begin{enumerate}
    \item \textbf{架构创新}:提出了结合 SignNet 和 Transformer 的混合架构,利用谱特征的全局性和尺寸不变性,为跨规模参数预测奠定了基础。
    
    \item \textbf{复杂度优势}:将量子测量复杂度从 $O(P^2)$ 降至 $O(1)$,仅引入 $O(N^3)$ 的经典预处理成本。这为 NISQ 设备上运行深度量子电路提供了可能。
    
    \item \textbf{跨规模泛化}:实验验证了在 $N \in [6,13]$ 上训练的模型可以零次迁移至 $N \in [14,28]$ 的更大系统,性能呈渐进衰减而非急剧崩溃。
    
    \item \textbf{噪声鲁棒性}:在中高噪声条件下,神经网络预测显著优于噪声累积的硬件执行,优势随噪声级别单调增加。
    
    \item \textbf{理论分析}:从参数集中现象和谱密度收敛理论两个角度,为神经网络的跨规模泛化能力提供了理论解释。
\end{enumerate}

\subsection{局限性}

\begin{enumerate}
    \item \textbf{振荡型样本}:对于正则图等导致持续振荡的系统,模型难以准确预测后段的高频振荡,这是数据驱动方法面对混沌动力学的固有挑战。
    
    \item \textbf{大规模验证}:由于经典模拟的指数级复杂度,我们无法为 $N > 20$ 的系统提供精确的 Ground Truth,跨规模实验依赖于合成轨迹和理论推断。
    
    \item \textbf{硬件验证}:本文的噪声模型是简化的,真实量子硬件上的性能有待验证。
\end{enumerate}

\subsection{未来工作}

\begin{enumerate}
    \item \textbf{张量网络验证}:利用矩阵乘积态(MPS)等张量网络方法,在 $N \sim 30-50$ 的一维或准一维系统上生成更可靠的测试数据。
    
    \item \textbf{物理信息微调}:引入可微量子模拟器,构建无需标签的物理损失函数,实现在大规模系统上的自监督训练。
    
    \item \textbf{混合预测策略}:对收敛型和振荡型样本采用不同的预测策略,或引入"包络线预测 + 细节填充"的两阶段方法。
    
    \item \textbf{真实硬件部署}:在 IBM、Google 等云量子平台上测试神经网络预测的参数,评估实际的优化性能和资源节省。
\end{enumerate}

\subsection{结语}

本文的核心贡献在于证明了:用经典计算(神经网络推理)替代昂贵的量子测量是可行且高效的。这种"经典大脑、量子身体"的混合范式——让经典计算机做它擅长的(模式识别和预测),让量子计算机做它擅长的(希尔伯特空间演化)——可能是通往实用量子优势的关键路径。随着量子硬件的成熟,在真实的 100+ 量子比特设备上部署这一框架,将是未来工作的重要方向。
本文提出了基于谱-时序 Transformer 的 FALQON 参数预测方法,通过"教师-学生"框架实现了量子优化参数的零次推理生成。主要发现包括:

\begin{enumerate}
    \item \textbf{可行性验证}:在收敛型样本上,模型达到 0.917 的平均相关系数,证明神经网络预测量子控制参数是可行的。
    \item \textbf{适用边界}:振荡型样本(主要来自正则图)的后段高频振荡难以准确预测,这是数据驱动方法面对混沌动力学的固有挑战。
    \item \textbf{理论保障}:从李雅普诺夫理论证明,只要预测误差不改变控制参数符号,系统仍能收敛。
\end{enumerate}

\paragraph{未来工作}
\begin{itemize}
    \item 在更大规模图($n > 20$)上验证跨规模泛化能力
    \item 探索混合策略:神经网络预测 + 物理约束细化
    \item 在真实量子硬件噪声模型下评估实际收益
\end{itemize}