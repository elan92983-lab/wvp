% 02_preliminaries.tex
% 预备知识

本节介绍 FALQON 算法的理论基础和图的谱表示,为后续方法论述奠定基础。

\subsection{FALQON 算法}
\label{subsec:falqon}

考虑组合优化问题的目标函数编码为问题哈密顿量 $H_P$,驱动哈密顿量取为横场 $H_D = \sum_{i=1}^n X_i$。FALQON 的目标是最小化成本函数:
\begin{equation}
C(t) = \langle \psi(t) | H_P | \psi(t) \rangle
\end{equation}

\begin{theorem}[FALQON 收敛性 \cite{magann2022lyapunov}]
\label{thm:falqon_convergence}
定义反馈控制律:
\begin{equation}
\beta(t) = -\alpha \cdot \langle \psi(t) | i[H_D, H_P] | \psi(t) \rangle, \quad \alpha > 0
\end{equation}
则成本函数满足 $\frac{dC}{dt} \leq 0$,即系统能量单调非递增。
\end{theorem}

\begin{proof}
根据薛定谔方程,成本函数的时间导数为:
\begin{equation}
\frac{dC}{dt} = i\beta(t) \langle \psi | [H_D, H_P] | \psi \rangle
\end{equation}
代入反馈律,由于 $\langle i[H_D, H_P] \rangle$ 为实数,得:
\begin{equation}
\frac{dC}{dt} = -\alpha \left( \langle i[H_D, H_P] \rangle \right)^2 \leq 0 \qquad \qed
\end{equation}
\end{proof}

在离散实现中,状态演化为:
\begin{equation}
|\psi_{p+1}\rangle = e^{-i\beta_p H_D} e^{-i H_P \Delta t} |\psi_p\rangle
\end{equation}
其中 $\beta_p = -\alpha \langle \psi_p | i[H_D, H_P] | \psi_p \rangle$。

\subsection{图的拉普拉斯谱}
\label{subsec:graph_spectrum}

给定无向图 $G = (V, E)$,其归一化拉普拉斯矩阵定义为:
\begin{equation}
L = I - D^{-1/2} A D^{-1/2}
\end{equation}
其中 $A$ 为邻接矩阵,$D$ 为度矩阵。$L$ 的特征分解 $L = U \Lambda U^T$ 给出:
\begin{itemize}
    \item \textbf{特征值} $\lambda_1 \leq \lambda_2 \leq \cdots \leq \lambda_n$:编码图的全局连通性
    \item \textbf{特征向量} $\{u_i\}_{i=1}^n$:提供节点的谱坐标
\end{itemize}

\paragraph{符号模糊性问题} 特征向量存在固有的符号歧义:若 $u$ 是 $L$ 的特征向量,则 $-u$ 也是。这对神经网络学习造成困难,我们采用 SignNet \cite{lim2023signnet} 解决此问题(详见 \ref{subsec:signnet} 节)。
