% 04_experiments.tex
% 实验部分

\subsection{实验设置}
\label{subsec:experimental_setup}

\subsubsection{数据集}

\paragraph{训练数据集}
训练数据集包含约 1000 个随机图样本,由两类图混合构成:
\begin{itemize}
    \item \textbf{Erdős-Rényi 图}(约 50\%):边概率 $p=0.5$,节点数 $n \in [6, 13]$
    \item \textbf{随机 3-正则图}(约 50\%):每节点度数固定为 3
\end{itemize}
每个样本包含 $P=40$ 层的 FALQON 参数序列,由经典模拟器以 $\alpha=1.0$ 生成。数据按 9:1 划分为训练集与测试集。

\paragraph{跨规模测试数据集}
为验证模型的量子比特数扩展性,我们生成了四组测试数据:
\begin{itemize}
    \item \textbf{域内}:$N \in [6, 13]$,100 个样本
    \item \textbf{轻度外推}:$N \in [14, 17]$,80 个样本
    \item \textbf{强外推}:$N \in [18, 22]$,60 个样本
    \item \textbf{极端外推}:$N \in [23, 28]$,40 个样本
\end{itemize}
对于 $N > 12$ 的大图,由于精确量子模拟的内存限制($2^N$ 维希尔伯特空间),我们使用基于谱特性的合成轨迹作为参考。

\paragraph{噪声测试数据集}
为评估噪声鲁棒性,我们在 $N \in [6, 10]$ 的小图上生成了五组不同噪声级别的数据:
\begin{itemize}
    \item \textbf{无噪声}:$\sigma_{\text{shot}}=0, \gamma=0, p_{\text{gate}}=0$
    \item \textbf{低噪声}:$\sigma_{\text{shot}}=0.05, \gamma=0.01, p_{\text{gate}}=0.001$
    \item \textbf{中等噪声}:$\sigma_{\text{shot}}=0.1, \gamma=0.02, p_{\text{gate}}=0.005$
    \item \textbf{高噪声}:$\sigma_{\text{shot}}=0.2, \gamma=0.05, p_{\text{gate}}=0.01$
    \item \textbf{极端噪声}:$\sigma_{\text{shot}}=0.3, \gamma=0.1, p_{\text{gate}}=0.02$
\end{itemize}
每组包含 50 个样本,同时记录干净轨迹和噪声轨迹。

\subsubsection{评估指标}

\begin{itemize}
    \item \textbf{皮尔逊相关系数 (Corr)}:$\text{Corr}(\hat{\bm{\beta}}, \bm{\beta}) \in [-1, 1]$,衡量预测与真实轨迹的趋势一致性
    \item \textbf{平均绝对误差 (MAE)}:$\frac{1}{P}\sum_t |\hat{\beta}_t - \beta_t|$
    \item \textbf{均方根误差 (RMSE)}:$\sqrt{\frac{1}{P}\sum_t (\hat{\beta}_t - \beta_t)^2}$
\end{itemize}

\subsubsection{样本分类标准}

根据参数序列后半段方差,将样本分为两类:
\begin{equation}
\text{样本类型} = \begin{cases}
\text{收敛型} & \text{if } \text{Var}(\beta_{P/2:P}) \leq 0.1 \\
\text{振荡型} & \text{otherwise}
\end{cases}
\end{equation}

\subsection{域内性能评估}
\label{subsec:in_domain_results}

表 \ref{tab:main_results} 展示了模型在域内测试集上的整体性能及分类表现。

\begin{itemize}
    \item \textbf{皮尔逊相关系数 (Corr)}:$\text{Corr}(\hat{\bm{\beta}}, \bm{\beta}) \in [-1, 1]$
    \item \textbf{平均绝对误差 (MAE)}:$\frac{1}{P}\sum_t |\hat{\beta}_t - \beta_t|$
    \item \textbf{均方根误差 (RMSE)}
\end{itemize}

\subsubsection{样本分类}

根据参数序列后半段方差,将样本分为两类:
\begin{equation}
\text{样本类型} = \begin{cases}
\text{收敛型} & \text{if } \text{Var}(\beta_{P/2:P}) \leq 0.1 \\
\text{振荡型} & \text{otherwise}
\end{cases}
\end{equation}

\subsection{主要结果}
\label{subsec:main_results}

表 \ref{tab:main_results} 展示了模型在测试集上的整体性能及分类表现。

\begin{table}[htbp]
\centering
\caption{谱-时序 Transformer 的测试集性能}
\label{tab:main_results}
\begin{tabular}{lcccc}
\toprule
样本类型 & 占比 & MAE $\downarrow$ & Corr $\uparrow$ & 最佳/最差 Corr \\
\midrule
收敛型 & 30\% & 0.215 & \textbf{0.917} & 0.997 / 0.804 \\
振荡型 & 70\% & 0.458 & 0.801 & 0.946 / 0.616 \\
\midrule
\textbf{总体} & 100\% & 0.314 & \textbf{0.885} & — \\
\bottomrule
\end{tabular}
\end{table}

\paragraph{关键发现}
\begin{enumerate}
    \item 模型在收敛型样本上表现优异(Corr = 0.917),最佳样本几乎完美拟合(Corr = 0.997)。
    \item 振荡型样本的后段高频振荡难以准确预测,但主要趋势仍被捕捉(Corr > 0.8)。
    \item 所有样本的初始"峰-谷"结构(Layer 0-5)均被准确拟合,这是决定优化方向的关键区间。
\end{enumerate}

\subsection{定性分析}
\label{subsec:qualitative}

图 \ref{fig:case_study} 展示了四个典型样本的预测结果。

\begin{figure}[htbp]
  \centering
  \includegraphics[width=0.9\linewidth]{figures/case_study.png}
  \caption{四个典型样本的预测结果示例(2x2 case study)}
  \label{fig:case_study}
\end{figure}

% 此处插入图片,需要准备对应的 figures 目录

\subsection{Cross-scale Generalization Results}
\label{subsec:cross_scale_results}

\begin{figure}[htbp]
  \centering
  \includegraphics[width=0.8\linewidth]{figures/corr_vs_n.png}
  \caption{Prediction correlation (Corr) vs. graph size $N$. Each point is a test sample; colors indicate graph type (ER or 3-regular).}
  \label{fig:scal_corr}
\end{figure}

\begin{figure}[htbp]
  \centering
  \includegraphics[width=0.7\linewidth]{figures/boxplot_by_range.png}
  \caption{Distribution of correlation coefficients for different scale ranges (In-domain, Mild/Strong/Extreme extrapolation). Boxes show median and IQR.}
  \label{fig:scal_box}
\end{figure}

\subsection{Noise Robustness Results}
\label{subsec:noise_results}

\begin{figure}[htbp]
  \centering
  \includegraphics[width=0.8\linewidth]{figures/noise_robustness_comparison.png}
  \caption{Comparison of correlation with the clean reference across noise levels: Neural Network (ours) vs. Noisy FALQON.}
  \label{fig:noise_comp}
\end{figure}

\begin{figure}[htbp]
  \centering
  \includegraphics[width=0.7\linewidth]{figures/nn_advantage_curve.png}
  \caption{Neural network advantage (Corr$_{NN}$ - Corr$_{NoisyFALQON}$) as a function of noise level.}
  \label{fig:noise_adv}
\end{figure}

\subsection{Spectral Density Analysis}
\label{subsec:spectral_analysis}

\begin{figure}[htbp]
  \centering
  \includegraphics[width=0.8\linewidth]{figures/spectral_density_by_n.png}
  \caption{Empirical spectral density evolution across different graph sizes (comparison with Wigner semicircle approximation).}
  \label{fig:spectral_density}
\end{figure}

\begin{figure}[htbp]
  \centering
  \includegraphics[width=0.8\linewidth]{figures/kl_divergence_matrix.png}
  \caption{KL divergence matrix between spectral histograms at different graph sizes.}
  \label{fig:kl_matrix}
\end{figure}

\begin{figure}[htbp]
  \centering
  \includegraphics[width=0.8\linewidth]{figures/convergence_to_semicircle.png}
  \caption{Wasserstein distance between empirical spectral density and semicircle law as a function of graph size.}
  \label{fig:convergence}
\end{figure}

\subsection{消融实验}
\label{subsec:ablation}

表 \ref{tab:ablation} 展示了各组件对模型性能的贡献。

\begin{table}[htbp]
\centering
\caption{消融实验结果}
\label{tab:ablation}
\begin{tabular}{lcc}
\toprule
配置 & Corr & $\Delta$ \\
\midrule
完整模型 & 0.885 & — \\
移除 SignNet & 0.821 & -0.064 \\
移除 Scheduled Sampling & 0.847 & -0.038 \\
移除时序梯度损失 & 0.869 & -0.016 \\
\bottomrule
\end{tabular}
\end{table}

\subsection{理论分析:预测误差的鲁棒性}
\label{subsec:robustness}

\begin{proposition}[预测误差的鲁棒性]
设学生模型预测 $\hat{\beta}_p = \beta_p + \epsilon_p$,若 $|\epsilon_p| < |\beta_p|$(即误差不改变符号),则能量仍单调下降。
\end{proposition}

这解释了为何即使存在预测误差,模型输出的参数序列仍能有效驱动优化。