% 05_related_work.tex
% 相关工作

\subsection{变分量子优化}

量子近似优化算法(QAOA)\cite{farhi2014qaoa} 是最广泛研究的变分量子算法,通过交替应用问题哈密顿量和混合哈密顿量实现组合优化。然而,QAOA 面临严重的参数优化困难,尤其是在大规模系统中的贫瘠高原问题。

近年来,研究者探索了多种参数初始化和迁移策略。Joshi 等人 \cite{joshi2022qaoa} 使用图神经网络预测 QAOA 参数,实现了跨实例迁移。Streif 和 Leib 提出了无需量子处理器的 QAOA 参数训练方法。本文的工作可视为将这一思路拓展至 FALQON 框架,并深入分析了跨规模泛化能力。

\subsection{基于反馈的量子控制}

FALQON \cite{magann2022feedback} 及其变体 \cite{magann2022lyapunov} 基于量子李雅普诺夫控制理论,提供了无需经典优化的量子控制方案。FALQON 保证了能量单调递减,从根本上避免了贫瘠高原问题。

然而,FALQON 的逐层测量机制导致了 $O(P^2)$ 的累积开销和噪声累积问题。本文的神经网络方法与 FALQON 形成互补:前者提供快速的零次推理,后者可用于在真实硬件上的精细微调。

\subsection{图的谱表示学习}

谱图神经网络利用拉普拉斯特征向量进行节点嵌入,在图分类和节点分類任務中取得了優異性能。然而,特征向量的符号模糊性一直是一个挑战。

SignNet \cite{lim2023signnet} 通过对称化操作解决了符号模糊性问题,在图级任务上显著提升了性能。本文首次将 SignNet 应用于量子优化参数预测,并验证了其在消除训练不稳定性方面的关键作用。

\subsection{物理信息神经网络}

物理信息神经网络(Physics-Informed Neural Networks, PINNs)将物理定律作为约束或损失函数引入神经网络训练,在科学计算中取得了广泛应用。

本文的方法可视为 PINN 在量子控制领域的应用:神经网络学习的是 FALQON 动力学的"典型模式",隐式编码了李雅普诺夫控制律的物理约束。未来工作可以探索显式引入物理损失函数进行自监督训练。
\subsection{变分量子优化}

QAOA \cite{farhi2014qaoa} 是最广泛研究的变分量子算法。近年来,研究者探索了多种参数初始化和迁移策略,包括基于图神经网络的参数预测 \cite{joshi2022qaoa}。本文的方法可视为将这一思路拓展至 FALQON 框架。

\subsection{基于反馈的量子控制}

FALQON \cite{magann2022feedback} 及其变体 \cite{magann2022lyapunov} 提供了无需经典优化的量子控制方案。本文的神经网络方法与 FALQON 互补:前者提供快速初始化,后者可用于微调。

\subsection{图的谱表示学习}

谱图神经网络利用拉普拉斯特征向量进行节点嵌入。SignNet \cite{lim2023signnet} 解决了特征向量的符号模糊性问题,本文将其首次应用于量子优化参数预测。
