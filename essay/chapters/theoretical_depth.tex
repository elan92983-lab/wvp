\section{提升理论深度:从经验拟合到物理定律}
为了满足用户“提升理论深度”的要求,本章节将经验性的实验结果提升到理论物理的高度,构建一个严谨的解释框架。

\subsection{量子李雅普诺夫控制(QLC)的收敛性分析}
FALQON 的理论根基是李雅普诺夫稳定性理论。这里明确具体的收敛性证明逻辑,并分析学生模型误差对收敛性的影响。

\textbf{定理(FALQON 收敛性):}
定义李雅普诺夫函数 $V(\boldsymbol{\beta}) = \langle \psi(\boldsymbol{\beta}) | H_P | \psi(\boldsymbol{\beta}) \rangle$。
根据薛定谔方程 $i\frac{\partial}{\partial t}|\psi\rangle = H(t)|\psi\rangle$,其中 $H(t) = H_P + \beta(t)H_d$(设 $\hbar=1$),我们有目标函数的导数:
\begin{equation}
\frac{d}{dt}\langle H_P \rangle = i \langle \psi | [H(t), H_P] | \psi \rangle = i \beta(t) \langle \psi | [H_d, H_P] | \psi \rangle
\end{equation}
设定反馈律 $\beta(t) = -\alpha \cdot i \langle [H_d, H_P] \rangle$(其中 $\alpha > 0$),由于 $H_d, H_P$ 均为厄米算符,其对易子的期望值为纯虚数,故 $i \langle [H_d, H_P] \rangle$ 为实数。代入后得:
\begin{equation}
\frac{d}{dt}\langle H_P \rangle = - \alpha \left( i \langle [H_d, H_P] \rangle \right)^2 \le 0
\end{equation}
这保证了能量随演化时间单调非递增,即系统必然流向 $H_P$ 的低能子空间。

\textbf{理论提升点:}
学生模型的预测值 $\hat{\beta} = \beta_{teacher} + \epsilon$。只要预测误差 $\epsilon$ 不足以改变反馈项的符号(Sign),即 $\text{sgn}(\hat{\beta}) = \text{sgn}(\beta_{teacher})$,则导数项 $\frac{d}{dt}\langle H_P \rangle$ 依然保持非正。这解释了为什么学生模型即使存在回归误差(MSE $> 0$),其最终生成的能量效果(AR)依然很高。
\textbf{结论:FALQON 对幅度误差具有鲁棒性,但对符号误差敏感}\cite{ref2,ref13}。这也进一步印证了解决谱特征符号歧义(Sign Ambiguity)的重要性。

\subsection{局部子图同构与参数可迁移性}
除了全局的 Kesten-McKay 解释外,还应引入基于局部子图的解释,这在 QAOA 文献中更为常见。
\begin{itemize}
    \item \textbf{光锥(Lightcone)原理}:在深度为 $p$ 的量子电路中,一个量子比特的可观测量的期望值仅取决于其在图上距离为 $p$ 以内的邻居节点。
    \item \textbf{树状近似(Tree-like Approximation)}:对于稀疏随机图,当 $N$ 很大时,几乎所有节点的局部 $p$-邻域都是一棵树(无环)。这意味着对于较小的 $p$,所有节点的局部环境在统计上是同构的(都是 $d$-正则树)\cite{ref12}。
    \item \textbf{推论}:因此,对于浅层电路,最优参数仅取决于度数 $d$,而与图的具体大小 $N$ 无关。这为“零次推理”提供了坚实的微观理论支撑:模型学习的是 $d$-正则树上的最优控制协议。只有当深度 $p$ 增大到足以“看到”图中的环(Loops)时,这种简单的迁移才会逐渐失效。
\end{itemize}
