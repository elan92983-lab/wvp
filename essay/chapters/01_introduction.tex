% 01_introduction.tex
% 引言部分

在嘈杂中型量子(NISQ)时代,变分量子算法(VQA)被广泛认为是通向实用量子优势的最可行路径之一。其中,量子近似优化算法(QAOA)\cite{farhi2014qaoa} 在解决组合优化问题(如 MaxCut、MaxSAT)方面展现出巨大潜力。然而,QAOA 的实际部署面临两大核心挑战:

\paragraph{挑战一:经典优化的困难}
QAOA 的参数优化需要在高维非凸能量景观中寻找最优参数 $\bm{\gamma}^*, \bm{\beta}^*$,极易陷入局部极小值。更严重的是,在大规模系统中存在"贫瘠高原"(Barren Plateau)现象——成本函数的梯度随量子比特数 $N$ 呈指数级衰减,使得基于梯度的优化方法完全失效。

\paragraph{挑战二:量子资源的可扩展性}
即使找到了优化策略,随着问题规模(量子比特数 $N$)的增加,所需的量子电路深度和测量次数也急剧增长。在当前的 NISQ 硬件上,相干时间有限,噪声严重,这使得大规模量子优化在实践中面临巨大障碍。

\subsection{FALQON:消除优化循环的代价}

基于反馈的量子优化算法(Feedback-based ALgorithm for Quantum OptimizatioN, FALQON)\cite{magann2022feedback} 提供了一种创新方案:利用量子李雅普诺夫控制理论,通过实时测量反馈自动确定每层参数,\textbf{完全消除经典优化循环}。FALQON 的控制律保证了系统能量单调递减,从而绕过了贫瘠高原问题。

然而,FALQON 引入了新的计算瓶颈——\textbf{累积测量开销}。为计算第 $p+1$ 层参数 $\beta_{p+1}$,必须先制备深度为 $p$ 的量子态并测量对易子期望值。对于深度为 $P$ 的电路,总累积电路深度达 $O(P^2)$。此外,在 NISQ 硬件上,每次测量都受到散粒噪声(Shot Noise)和退相干(Decoherence)的影响,这些噪声会通过反馈机制逐层累积,导致控制轨迹严重偏离理想路径。

\subsection{本文的核心思想与贡献}

本文提出一种"教师-学生"零次推理框架,核心思想是:
\begin{quote}
\textit{用神经网络一次性预测完整的控制参数序列 $\{\beta_t\}_{t=0}^{P-1}$,从而将 $O(P^2)$ 的累积测量开销降为 $O(1)$ 的单次电路执行(仅用于最终验证)。}
\end{quote}

本文不仅关注\textbf{电路深度}维度的复杂度优化($O(P^2) \to O(1)$),更深入探讨了\textbf{量子比特数}维度的可扩展性——即在小规模系统上训练的模型能否迁移到大规模系统。此外,我们分析了神经网络预测相对于噪声硬件执行的\textbf{鲁棒性优势}。

具体而言,本文的主要贡献包括:

\begin{enumerate}
    \item \textbf{架构设计}:提出谱-时序 Transformer(Spectral-Temporal Transformer),结合符号不变网络(SignNet)\cite{lim2023signnet} 处理图的拉普拉斯谱特征,利用 Transformer 解码器捕捉参数序列的时序依赖。
    
    \item \textbf{跨规模泛化分析}:基于参数集中现象(Parameter Concentration)和谱密度收敛理论,从理论和实验两方面论证了模型的量子比特数扩展性——在 $N \in [6,13]$ 上训练的模型可以零次迁移至 $N \in [14,28]$ 的更大系统。
    
    \item \textbf{噪声鲁棒性实验}:系统性地比较了神经网络预测与噪声硬件执行 FALQON 的性能,证明在中高噪声条件下,神经网络预测的轨迹质量显著优于噪声累积的硬件执行。
    
    \item \textbf{复杂度分析}:给出了完整的量子/经典复杂度权衡分析,证明用 $O(N^3)$ 的经典预处理换取 $O(1)$ 的量子测量是高度划算的策略。
\end{enumerate}

具体而言,本文的主要贡献包括:
\begin{enumerate}
    \item \textbf{架构设计}:提出谱-时序 Transformer(Spectral-Temporal Transformer),结合符号不变网络(SignNet)\cite{lim2023signnet} 处理图的拉普拉斯谱特征,利用 Transformer 解码器捕捉参数序列的时序依赖。
    
    \item \textbf{训练策略}:引入 Scheduled Sampling \cite{bengio2015scheduled} 缓解自回归模型的训练-推理不一致问题,并设计针对难样本的加权损失函数。
    
    \item \textbf{系统性评估}:在包含 1000 个随机图的数据集上进行全面实验,\textbf{首次}按动力学特性将样本分类为"收敛型"与"振荡型",揭示了模型的适用边界。
    
    \item \textbf{理论分析}:从李雅普诺夫稳定性角度证明,只要预测误差不改变控制参数的符号,系统仍能收敛至低能态,为神经网络预测的鲁棒性提供理论保障。
\end{enumerate}
