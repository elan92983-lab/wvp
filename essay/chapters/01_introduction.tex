% 01_introduction.tex
% 引言部分 - 不包含 \section 命令,由 main.tex 统一管理

在嘈杂中型量子(NISQ)时代,变分量子算法(VQA)被广泛认为是通向实用量子优势的最可行路径之一。其中,量子近似优化算法(QAOA)\cite{farhi2014qaoa} 在解决组合优化问题(如 MaxCut、MaxSAT)方面展现出巨大潜力。然而,QAOA 的实际部署面临严峻挑战:经典参数优化循环需要在高维非凸能量景观中寻找最优参数 $\bm{\gamma}^*, \bm{\beta}^*$,极易陷入局部极小值,且在大规模系统中面临"贫瘠高原"(Barren Plateau)现象。

基于反馈的量子优化算法(Feedback-based ALgorithm for Quantum OptimizatioN, FALQON)\cite{magann2022feedback} 提供了一种替代方案:利用量子李雅普诺夫控制理论,通过实时测量反馈自动确定每层参数,从而\textbf{完全消除经典优化循环}。然而,FALQON 引入了新的计算瓶颈——\textbf{测量开销}:为计算第 $p+1$ 层参数 $\beta_{p+1}$,必须先制备深度为 $p$ 的量子态并测量对易子期望值,导致总累积电路深度达 $O(P^2)$。

\paragraph{本文贡献} 我们提出一种"教师-学生"零次推理框架,核心思想是:
\begin{quote}
\textit{用神经网络一次性预测完整的控制参数序列 $\{\beta_t\}_{t=0}^{P-1}$,从而将 $O(P^2)$ 的累积测量开销降为 $O(P)$ 的单次电路执行。}
\end{quote}

具体而言,本文的主要贡献包括:
\begin{enumerate}
    \item \textbf{架构设计}:提出谱-时序 Transformer(Spectral-Temporal Transformer),结合符号不变网络(SignNet)\cite{lim2023signnet} 处理图的拉普拉斯谱特征,利用 Transformer 解码器捕捉参数序列的时序依赖。
    
    \item \textbf{训练策略}:引入 Scheduled Sampling \cite{bengio2015scheduled} 缓解自回归模型的训练-推理不一致问题,并设计针对难样本的加权损失函数。
    
    \item \textbf{系统性评估}:在包含 1000 个随机图的数据集上进行全面实验,\textbf{首次}按动力学特性将样本分类为"收敛型"与"振荡型",揭示了模型的适用边界。
    
    \item \textbf{理论分析}:从李雅普诺夫稳定性角度证明,只要预测误差不改变控制参数的符号,系统仍能收敛至低能态,为神经网络预测的鲁棒性提供理论保障。
\end{enumerate}
