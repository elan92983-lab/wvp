% 04_experiments.tex (English)

\subsection{Experimental Setup}
\subsubsection{Dataset}
We use 1000 random graphs composed of two families: Erd\H{o}s-R\'enyi (p=0.6, n in [6,13]) and random 3-regular graphs. Each sample contains a FALQON parameter sequence of length $P=40$ generated by a classical simulator with $\alpha=1.0$. Data are split 90/10 train/test.

\subsubsection{Metrics}
We report Pearson correlation (Corr), MAE, and RMSE.

\subsubsection{Sample Categorization}
We classify samples as convergent if $\mathrm{Var}(\beta_{P/2:P})\le 0.1$, otherwise oscillatory.

\subsection{Main Results}
Table \ref{tab:main_results} summarizes performance.

\begin{table}[htbp]
\centering
\caption{Test performance of the Spectral-Temporal Transformer}
\label{tab:main_results}
\begin{tabular}{lcccc}
\toprule
Type & Share & MAE $\downarrow$ & Corr $\uparrow$ & Best/Worst Corr \\
\midrule
Convergent & 30\% & 0.215 & \textbf{0.917} & 0.997 / 0.804 \\
Oscillatory & 70\% & 0.458 & 0.801 & 0.946 / 0.616 \\
\midrule
\textbf{Overall} & 100\% & 0.314 & \textbf{0.885} & — \\
\bottomrule
\end{tabular}
\end{table}

\subsection{Qualitative Analysis}
Figure \ref{fig:case_study} shows four representative predictions.

\begin{figure}[htbp]
  \centering
  \includegraphics[width=0.9\linewidth]{../essay/figures/case_study.png}
  \caption{Case study: four representative test samples (2x2 montage).}
  \label{fig:case_study}
\end{figure}

\subsection{Cross-scale Generalization Results}
\label{subsec:cross_scale_results}

\begin{figure}[htbp]
  \centering
  \includegraphics[width=0.8\linewidth]{../essay/figures/corr_vs_n.png}
  \caption{Prediction correlation (Corr) vs. graph size $N$. Each point is a test sample; colors indicate graph type (ER or 3-regular).}
  \label{fig:scal_corr}
\end{figure}

\begin{figure}[htbp]
  \centering
  \includegraphics[width=0.7\linewidth]{../essay/figures/boxplot_by_range.png}
  \caption{Distribution of correlation coefficients for different scale ranges (In-domain, Mild/Strong/Extreme extrapolation). Boxes show median and IQR.}
  \label{fig:scal_box}
\end{figure}

\subsection{Noise Robustness Results}
\label{subsec:noise_results}

\begin{figure}[htbp]
  \centering
  \includegraphics[width=0.8\linewidth]{../essay/figures/noise_robustness_comparison.png}
  \caption{Comparison of correlation with the clean reference across noise levels: Neural Network (ours) vs. Noisy FALQON.}
  \label{fig:noise_comp}
\end{figure}

\begin{figure}[htbp]
  \centering
  \includegraphics[width=0.7\linewidth]{../essay/figures/nn_advantage_curve.png}
  \caption{Neural network advantage (Corr$_{NN}$ - Corr$_{NoisyFALQON}$) as a function of noise level.}
  \label{fig:noise_adv}
\end{figure}

\subsection{Spectral Density Analysis}
\label{subsec:spectral_analysis}

\begin{figure}[htbp]
  \centering
  \includegraphics[width=0.8\linewidth]{../essay/figures/spectral_density_by_n.png}
  \caption{Empirical spectral density evolution across different graph sizes (comparison with Wigner semicircle approximation).}
  \label{fig:spectral_density}
\end{figure}

\begin{figure}[htbp]
  \centering
  \includegraphics[width=0.8\linewidth]{../essay/figures/kl_divergence_matrix.png}
  \caption{KL divergence matrix between spectral histograms at different graph sizes.}
  \label{fig:kl_matrix}
\end{figure}

\begin{figure}[htbp]
  \centering
  \includegraphics[width=0.8\linewidth]{../essay/figures/convergence_to_semicircle.png}
  \caption{Wasserstein distance between empirical spectral density and semicircle law as a function of graph size.}
  \label{fig:convergence}
\end{figure}

\subsection{Ablation}
Table \ref{tab:ablation} shows the contribution of components.

\begin{table}[htbp]
\centering
\caption{Ablation study results}
\label{tab:ablation}
\begin{tabular}{lcc}
\toprule
Configuration & Corr & $\Delta$ \\
\midrule
Full model & 0.885 & — \\
Remove SignNet & 0.821 & -0.064 \\
Remove Scheduled Sampling & 0.847 & -0.038 \\
Remove temporal gradient loss & 0.869 & -0.016 \\
\bottomrule
\end{tabular}
\end{table}
